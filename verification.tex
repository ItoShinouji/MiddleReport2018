\section{戦略の検証}

Thorp氏によって考案されたベーシックストラテジーについて検証を行う.
ベーシックストラテジーの性能評価のために比較対象として6つの戦略を考えた.
比較対象となる戦略は次の通りである.

\begin{itemize}
  \item ベーシックストラテジー改変1
  \item ベーシックストラテジー改変2
  \item プレイヤーの合計値が15以上になるまでヒット
  \item プレイヤーの合計値が16以上になるまでヒット
  \item プレイヤーの合計値が17以上になるまでヒット
  \item プレイヤーの合計値が18以上になるまでヒット
\end{itemize}

以上の6つの戦略について,これから一つ一つ詳細に説明する.
\bunseki{※米村祥裕}

\subsection{ベーシックストラテジー改変1}
ベーシックストラテジーはブラックジャックにおける有効な戦略の一つである.しかし
戦略の表に注目すると,表の複雑性を考えたときに変更の余地があると考えた.
戦略表においてプレイヤーの合計値が12の行に注目する.ディーラーのアップカードが2,3の時は
Sとなっているが,その後アップカードが4,5,6の時はH,アップカードが7,8,9,10,Aの時はHとなっており,
Hに挟まれてSが存在している.表を覚えることを考えると,同じ行の中で変化が少ない方がよいと考えられる.
そのため,ベーシックストラテジー改変1では表\ref{bschange1}に示すように,プレイヤーの合計値が12,ディーラーのアップ
カードが4,5,6の時の戦略をSに変更した.

\begin{table}[htbp]
  \centering
  \caption{ベーシックストラテジー改変1\label{bschange1}}
  \begin{tabular}{|c|c|c|c|c|c|c|c|c|c|c|c|}
    \hline
    \multicolumn{2}{|c|}{} & \multicolumn{10}{|c|}{ディーラーのアップカード} \\ \hline
    \multicolumn{2}{|c|}{} & 2 & 3 & 4 & 5 & 6 & 7 & 8 & 9 & 10 & A \\ \hline
    手札の合計 & 19以上 & S & S & S & S & S & S & S & S & S & S \\ \cline{3-12}
              & 18 & S & S & S & S & S & S & S & S & S & S \\ \cline{3-12}
              & 17 & S & S & S & S & S & S & S & S & S & S \\ \cline{3-12}
              & 16 & S & S & S & S & S & H & H & H & H & H \\ \cline{3-12}
              & 15 & S & S & S & S & S & H & H & H & H & H \\ \cline{3-12}
              & 13\~ 14 & S & S & S & S & S & H & H & H & H & H \\ \cline{3-12}
              & 12 & S & S & S & S & S & H & H & H & H & H \\ \cline{3-12}
              & 11以下 & H & H & H & H & H & H & H & H & H & H \\ \hline
  \end{tabular}
\end{table}

\subsection{ベーシックストラテジー改変2}

\begin{table}[htbp]
  \centering
  \caption{ベーシックストラテジー改変2\label{bschange2}}
  \begin{tabular}{|c|c|c|c|c|c|c|c|c|c|c|c|}
    \hline
    \multicolumn{2}{|c|}{} & \multicolumn{10}{|c|}{ディーラーのアップカード} \\ \hline
    \multicolumn{2}{|c|}{} & 2 & 3 & 4 & 5 & 6 & 7 & 8 & 9 & 10 & A \\ \hline
    手札の合計 & 19以上 & S & S & S & S & S & S & S & S & S & S \\ \cline{3-12}
              & 18 & S & S & S & S & S & S & S & S & S & S \\ \cline{3-12}
              & 17 & S & S & S & S & S & S & S & S & S & S \\ \cline{3-12}
              & 16 & S & S & S & S & S & H & H & H & H & H \\ \cline{3-12}
              & 15 & S & S & S & S & S & H & H & H & H & H \\ \cline{3-12}
              & 13\~ 14 & S & S & S & S & S & H & H & H & H & H \\ \cline{3-12}
              & 12 & H & H & H & H & H & H & H & H & H & H \\ \cline{3-12}
              & 11以下 & H & H & H & H & H & H & H & H & H & H \\ \hline
  \end{tabular}
\end{table}

\subsection{プレイヤーの合計値が15以上になるまでヒット}
\subsection{プレイヤーの合計値が16以上になるまでヒット}
\subsection{プレイヤーの合計値が17以上になるまでヒット}
\subsection{プレイヤーの合計値が18以上になるまでヒット}