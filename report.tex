% プロジェクト学習中間報告書書式テンプレート ver.1.0 (iso-2022-jp)

% 両面印刷する場合は `openany' を削除する
\documentclass[11pt,a4paper,oneside]{jsbook}

\usepackage{bm}
\usepackage{float}
\usepackage{amsmath}
\usepackage{amsfonts}
\usepackage{amssymb}
\usepackage{blindtext}
\usepackage{here}
\usepackage[sectionbib]{chapterbib}
\usepackage[dvipdfmx]{graphicx}
\usepackage{funpro}

\setcounter{chapter}{1}
\setcounter{section}{0}

\thisYear{2018}
\jProjectName{ディーラーをやっつけろ! 複雑系の数理とシミュレーション}
\eProjectName{Beat the dealer! Mathematics of Complex Systems and Simulation.}
\ProjectNumber{3}
\jGroupName{グループ~1}
\eGroupName{Group~1}
\ProjectLeader{1014207}{菱田美紗紀}{Misaki~Hishida}
\GroupLeader  {1016123}{薩田凱斗}{Kaito~Satta}
\SumOfMembers{9}
\GroupMember  {1}{1016007}{柏田輝}{Hikaru~Kashiwada}
\GroupMember  {2}{1016042}{尾崎拓海}{Takumi~Ozaki}
\GroupMember  {3}{1016078}{伊藤晋之介}{Shinnosuke~Ito}
\GroupMember  {4}{1016087}{驫木文弥}{Fumiya~Todoroki}
\GroupMember  {5}{1016118}{葛西隼人}{Hayato~Kasai}
\GroupMember  {6}{1016175}{柿崎大輝}{Daiki~Kakizaki}
\GroupMember  {7}{1016184}{鳥谷航大}{Koudai~Toriya}
\GroupMember  {8}{1016207}{渡邊凛}{Rin~Watanabe}
\GroupMember  {9}{1016231}{米村祥裕}{Yonemura~Yoshihiro}
\jadvisor{川越敏司,川口聡,斎藤朝輝}
\eadvisor{Toshiji~Kawagoe,Satoshi~Kawaguchi,Asaki~Saitou}
\jdate{2018年7月22日}
\edate{July~22, 2018}

\renewcommand{\thechapter}[1]{{\huge 第\arabic{chapter}章\quad #1}}%
\renewcommand{\thesection}{\arabic{chapter}.\arabic{section}}%

% 画像ファイル (EPS, EPDF, PNG) を読み込むために
\usepackage[dvipdfmx]{graphicx,color}
\pagestyle{empty}
\advance\textheight\headheight \headheight=0pt
\advance\textheight\headsep    \headsep=0pt
\advance\textheight\footskip   \footskip=0pt
\textheight=738truept
\advance\textwidth\marginparsep \marginparsep=0pt
\advance\textwidth\marginparwidth \marginparwidth=0pt
\advance\textwidth\oddsidemargin \oddsidemargin=0pt
\evensidemargin=\oddsidemargin
\textwidth=50zw
\advance\textwidth2zw
\columnsep=2zw
\topmargin=-5.4mm
\oddsidemargin=-7.4mm

\begin{document}
\maketitle
\pagenumbering{roman}
\fontsize{10}{18}\selectfont
%前付け
\frontmatter

% 和文概要
\begin{jabstract} 
\ \ 本プロジェクトでは、カジノにおいて最もポピュラーなゲームの一つであるブラックジャックを取り扱っている。ブラックジャックに関する代表的な行動戦略としてベーシックストラテジー、カウンティングというものがある。これらはプレイヤーの期待利得を増大させプレイヤーにとって有利な状況を作り出した。しかしカジノ側の対策によって、戦略を単純に採用し実行することは困難になり、期待利得も減少することになった。そこで我々は既存の戦略と比較して 3つの観点から優れた戦略を見つけ出すことを試みた。1つ目はプレイヤーの利得を既存の戦略に比べて増加させること、2つ目はカジノ側の対策を回避することである。3つ目はプレイヤーが覚えやすく、実行可能なものであることである。\\
\ \ 戦略を見つけるにあたって、現在までに我々は既存の戦略として最も有名なベーシックストラテジーと呼ばれる戦略についてどの程度確からしく、単純に考えられる他の手法に比べてどの程度プレイヤーに有利なのかを検証した。ブラックジャックのシミュレータを作成し、ゲームを行った。その勝率により戦略の優劣を比較した。このとき、戦略の複雑性を考慮するとベーシックストラテジーは改善の余地があるという結果が得られた。
% 和文キーワード
\begin{jkeyword}
カジノ,\ ブラックジャック,\ ベーシックストラテジー,\ 複雑性
\end{jkeyword}
\bunseki{※轟木文弥}
\end{jabstract}

%英語の概要
\begin{eabstract} 
\ \ In this project, we deal with Blackjack, one of the most popular games in the casino. There are also basic strategies and counting as representative behavior strategies concerning blackjack. This has increased the player's expectation gain and created a situation that is also true for the player. It is difficult to easily adopt and execute the strategy by the casino side measures, and the expected value has also been reduced. Then, we tried trying to find a winning strategy from three perspectives compared to existing strategy: one to increase the gain of the target player compared to the existing strategy, two to casino side The third is to make it easy for players to memorize and execute.\\
\ \ In finding a strategy, we have verified to what extent the strategy called the basic strategy which is most familiar as an existing strategy is confirmed up to now and to what extent it is advantageous to the player compared with other methods which can simply be considered. We made a blackjack simulator and played a game. Based on the winning percentage, we compared the merits of strategies. At this time, considering the complexity of the strategy, the result that the basic strategy has room for improvement was obtained.
% 英文キーワード
\begin{ekeyword}
casino, blackjack, basic strategies, complexity
\end{ekeyword}
\bunseki{※轟木文弥}
\end{eabstract}

\tableofcontents
\newpage

\pagenumbering{arabic}

%研究背景に関する項目
%\chapter{研究背景}

%\thechapter{課題解決のプロセス}
\thechapter{研究背景}
\chapter{はじめに}
\section{背景}
この章では、ブラックジャックの歴史やそれについて今まで研究されてきた戦略、またそれらの戦略の問題点を述べる。
\bunseki{※菱田美紗紀}

\subsection{ブラックジャックの概要と歴史}
ブラックジャックのルーツは1570年にさかのぼる。このころはまだ「ブラックジャック」という名称は使われていなかった。1875年に出版された「The American Hoyle of 1875」という書籍で「ブラックジャック」として紹介されたのが初出である。このゲームが考え出された当初は金銭を賭けることはなく、身内内で楽しむだけのゲームであった。
はじめて金銭が賭けられるようになったのは1910年頃のアメリカ、インディアナ州であるといわれている。当時は競馬以外のギャンブルが違法であり、ブラックジャックも違法カジノでのみ行われていた。その後インディアナ州以外のアメリカ全土に広がっていった。そして現在世界中のカジノで合法的に楽しまれるポピュラーなゲームになった。
\bunseki{※菱田美紗紀}

\subsection{ブラックジャックの戦略の歴史}
ブラックジャックについては多くの人間が戦略を考え出しては、カジノ側に禁止をされてきた歴史がある。1950年に、メリーランド州のとある米国陸軍の研究所に所属していた、Roger Nash Baldwin氏らが研究し始めたのが、ブラックジャックの戦略の研究のはじまりであるといわれている。その後パソコンが発明されたことにより、シミュレーションが容易になったことでさらに戦略の研究は進んでいった。
ブラックジャックには主に有名な戦略が2つ存在する。ベーシックストラテジーと呼ばれる、ディーラーのアップカードと自分の手札によって戦略を決定するものと、カウンティングと呼ばれる、今まで出たカードを記憶して戦略を決定するものである。
まず、ベーシックストラテジーについて詳しく説明する。この戦略は、1962年、カリフォルニア大学アーバイン校の教授である、Edward Thorp氏が、書籍「Beat The Dealer:A Winning Strategy for the Game of Twenty One」にて発表した。この戦略は、ディーラーのアップカード、自分の初期の手札の2つの情報から、次に自分がどのような行動を行えばよいのか決定される戦略である。この戦略を使用すれば勝率は4割~5割弱を出すことができる。
カウンティングについては後期に詳しく調査する予定である。
なお、いずれの戦略もカジノでは禁止されており、これらの戦略を使用していると気づかれたとき、プレイヤーはカジノから追放される。
\bunseki{※菱田美紗紀}

\subsection{従来の戦略の問題点}
まず、ベーシックストラテジーの問題点について述べる。この戦略はデック数が無限個を想定しており、すでに引いたカードの種類や枚数を考慮していない点である。
まず、基本的にカジノで行われるゲームではデック数は有限である。デック数が有限であるということは、以前に引いたカードの種類によって、次に引くカードの確率が変化していくということである。例えば、プレイヤーが2人、デック数が1のゲームをしていると仮定する。ディーラーのアップカードがハートのエース、プレイヤー1の初期の手札がダイヤのエースとクローバーのエース、プレイヤー2の初期の手札がスペードのエースとハートの2が配られたとする。この時点でデック内のエースのカードはすべて配られてしまい、以後のゲームでエースが出る確率は0になっている。しかし、デック数を無限に設定すると、以後のゲームでもエースは無限に存在することになり、確率は4/52になってしまう。これが、デック数無限のシミュレーションと実際のゲームの相違点である。このような相違点を考慮しないため、実際の勝率とは異なってしまい、利益についてもばらつきが出てしまうのである。
またこの戦略は勝率が4割程度しかなく、勝利数だけで言えばカジノ側に負けてしまう。
カウンティングの問題点については後期に詳しく調査する予定である。
\bunseki{※菱田美紗紀}

\section{ルール}
\subsection{ブラックジャック}
ブラックジャックとは、トランプを使用したゲームで、カジノで行われる有名なギャンブルの一つである。ブラックジャックはプレイヤーとディーラー(カジノ側のプレイヤー)が戦うゲームで、勝敗はトランプの合計で決まり、合計が21以下で相手より大きい方の勝利である。また、ブラックジャックを極めることが出来れば全てのカジノで勝ち易くなると言われており、カジノでの勝率を上げる手段としても注目されている。

\subsection{ブラックジャックのルール}
トランプの扱いについて
\begin{itemize}
\item ジョーカーは扱わない
\item トランプのマークは関係がない
\item 数字は2~9まではそのままの数
\item 10・J・Q・Kは10として扱う
\item Aは11または1で都合の良いほうとして扱う
\end{itemize}
勝敗条件
\begin{itemize}
\item 相手よりも合計が大きく、22より小さい方が勝ち
\item 22以上であるとバーストといい、バーストした側の負けとなる
\item 合計が同じ場合は引き分けとなる
\item プレイヤーとディーラーの両方がバーストの場合、ディーラーの勝ちとなる
\item 2枚の手札で合計が21になるとナチュラルブラックジャックといい、3枚以上の合計21と対決した場合、ナチュラルブラックジャックの方が勝ちになる
\end{itemize}
賭け金の扱い
\begin{itemize}
\item プレイヤーが普通に勝つと賭け金の2倍が払い戻される
\item プレイヤーがナチュラルブラックジャックで勝つと賭け金の2.5倍が払い戻される
\item ディーラーが勝つと賭け金を没収される
\end{itemize}
プレイヤーの選択肢
\begin{itemize}
\item ヒット:カードを1枚追加すること、何度でもできる
\item スタンド:カードを引かずに今のカードで勝負すること
\item サレンダー:負けを認めて、賭け金の半分をもらうことができる。最初の行動でのみ使える。
\item ダブルダウン:賭け金を2倍にし、1度だけヒットをする。最初の行動でのみ使える。
\item スプリット:最初のカード2枚が同じ数字だった場合使用可能。最初の賭け金と同じ金額を追加して、それを2つに分割して、それぞれで勝負することができる
\item インシュランス:ディーラーの表向きのカード(アップカード)が「A」の場合使える。最初の賭け金の半分を使い、ディーラーがナチュラルブラックジャックになればその賭け金の2倍が払い戻される。
\item イーブンマネー:自分の手札がナチュラルブラックジャックである場合に行うインシュランスのこと。このイーブンマネーの場合は元の賭金の半額をわざわざテーブルに出す必要はなく、ただ単にイーブンマネーと声を出して宣言するだけでよい。宣言するとすぐその場でディーラーは元の賭金と同じ額だけ支払ってくれる。なぜならディーラーにブラックジャックが完成していようがいまいが結果は必ず同じになるからだ。もしディーラーにブラックジャックが完成していた場合、インシュランスとして掛けた保険料の倍の金額が支払われ、もともとの勝負の方はお互いブラックジャックなので引き分け。結果として保険金だけを受け取ることになる。逆にもしディーラーにブラックジャックができていなかった場合、当然保険料は没収されるが、一方ゲームそのものの勝負はプレイヤー側の勝ちとなり賭金の1.5倍の勝ち金を受けることになる。つまり結果として差し引き 「賭金と同じ額だけの儲け」 ということになる。以上のようにイーブンマネー保険を掛けた場合はディーラーの見えていない方のカードに関係なく自動的に賭金の1倍の勝ちが確定することになる。よってイーブンマネーの場合はイーブンマネーと宣言するだけで良い。
\end{itemize}

\subsection{ディーラーの行動}
プレイヤーに比べ、ディーラーが選択できる行動は少なく、ヒットとスタンドしかできない。その上、ディーラーの行動はあるルールが存在する。そのルールとは「ディーラーは17以上になるまでヒットを続けなければならない」というルールだ。これによりディーラーの最終合計は17,18,19,20,21,バーストのどれかになる。

\subsection{ゲームの流れ}
まず山札をシャッフルし、カットカードと呼ばれるカードを山札にランダムに入れる。カーットカードは、ゲームが終わりを示すカードである。その後ゲームが始まり、まず賭け金を出す。その後、プレイヤー、ディーラーの順にカードが1枚配られ、再度同じように2枚目が配られます。このときプレイヤーのカードは2枚とも表向きですがディーラーは1枚を表向き(アップカード)でもう1枚は裏向き(ダウンカード)です。カードが配り終えると、プレイヤーの行動です。プレイヤーがバースト、もしくはスタンドした場合、プレイヤーの行動は終了である。次はディーラーの行動です。ディーラーがスタンド、もしくはバーストした場合、ディーラーの行動は終了である。ここで勝敗を確認し、それに応じた支払いが行われて、ゲームが終了する。
    \section{ブラックジャックにおける既存の戦略}
    \subsection{既存の戦略}

        ブラックジャックにおける主な既存の戦略はベーシックストラテジーとカウンティングである。

        ベーシックストラテジーは一回ごとのゲームを想定しており、自分の手札とアップカードのみによって勝率が高い行動を決定する戦略である。そのため、行動を決定するまでに使用されたカードは行動決定に影響せず、残りのカードの予測も行わない。

        また、カウンティングは、ゲームを複数回行ったことで発生する利得を最大にすることを目的とした戦略である。ゲームで使われたカードを記憶し、残りのカードを予測する。そこから自分の有利・不利を決定し掛け金の増減を行うという戦略である。
        \bunseki{※鳥谷 航大}
    \subsection{ベーシックストラテジー}
        ブラックジャックの最も有名な既存の戦略としてベーシックストラテジーという戦略が挙げられる。ベーシックストラテジーは1962年にEdward Thorp氏の書籍『Beat The Dealer: A Winning Strategy for the Game of Twenty One』にて発表された戦略である。元となるアイデアとして、Roger Baldwinらが1956年に発表した「The Optimum Strategy in Blackjack」が使用されている。
        \bunseki{※鳥谷 航大}
    \subsection{ベーシックストラテジーの行動決定式}
        ベーシックストラテジーは以下の前提条件を持つ。
        \begin{itemize}
            \item 前提条件\\
                どのカードを引く確率も13分の1である。
        \end{itemize}

        この前提条件の元で、アップカードと手札の合計値からヒットした時とスタンドした時の勝率から導出される。具体的には以下のような手順によって導出する。
        \begin{enumerate}
            \item プレイヤーの手札の合計値とアップカードの組み合わせごとにヒットした場合の勝率とスタンドした場合の勝率を求める
            \item ヒットした場合の勝率からスタンドした場合の勝率を引き、その差を求める
            \item 求めた差が0以上であった場合はヒット、0未満であった場合はスタンドが有効であるとする
        \end{enumerate}

        ベーシックストラテジーの導出に移る前に以下のような定義を行う。
        \begin{itemize}
            \item[] x: プレイヤーに最初に配られた2枚のカードの合計値 \((x < 21)\)
            \item[] D: アップカード
            \item[] M(D): ディーラーのアップカードによるプレイヤーがスタンドできる最低の手札の合計値
            \item[] J: プレイヤーが1回ヒットした後の最終的な手札の合計値
            \item[] T: ディーラーの最終的な手札の合計値 (\(T \geq 17\))
            \item[] \(E_{d,x}\): プレイヤーの手札の合計値がxの時にヒットした場合の勝率
            \item[] \(E_{s,x}\): プレイヤーの手札の合計値がxの時にスタンドした場合の勝率
            \item[] \(P(Y)\): 式Yが成立する確率
        \end{itemize}

        まず\(E_{s,x}\)について考える。\(E_{s,x}\)とはプレイヤーの手札の合計値がxである時にスタンドした場合の勝率である。つまり、\(T > 21 \)または\(T < x\)の時、プレイヤーはxでスタンドした場合、そのゲームに勝利し、\(x < T \leq 21\)の時、プレイヤーはxでスタンドした場合、そのゲームに敗北する。また、\(T = x\)の時は引き分けとなり、xでスタンドした場合利得は0である。これらのことから\(E_{s,x}\)は以下のような式によって表せる。
        \begin{displaymath}
            \begin{split}
                E_{s,x} = &P(T > 21) + P(T < x) - P(x < T \leq 21)\\
                &= 2P(T > 21) + 2P(T < x) + P(T = x) - 1
            \end{split}
        \end{displaymath}
        
        次に\(E_{d,x}\)について考える。\(E_{d,x}\)とはプレイヤーの手札の合計値がxである時にヒットした場合の勝率である。手札の合計値がxでヒットした場合は以下の3つの場合に分けられる。
        \begin{enumerate}
            \item \(J < 17\)\\
                Tは常に17以上であるから、この時にプレイヤーが勝利する期待値は以下のように表せる。
                $$P(T > 21) - (1 - P(T > 21)) = 2P(T > 21) - 1$$
            \item \(17 \leq J \leq 21\)\\
                同様にこの時にプレイヤーが勝利する期待値は、
                $$P(T > 21) + P(T < J) - P(J < T \leq 21)$$
                となる。
            \item \(J > 21\)\\
                この時、Tがどのような値でもプレイヤーは敗北する。つまり期待値は-1となる。
        \end{enumerate}

        以上のことから\(E_{d,x}\)は、
        \begin{displaymath}
            \begin{split}
                E_{d,x} = &P(J < 17)(2P(T > 21) - 1) - P(J > 21)\\
                &+ \sum_{j=17}^{21}P(J = j)(P(T > 21) + P(T < j) - P(j < T \leq 21))
            \end{split}
        \end{displaymath}
        となる。
        ここで\(E_{d,x}\)の第3項の部分に注目する。第3項は、Tは常に17以上\((T > 17)\)であること、Jは17以上21以下\((17 \leq J \leq 21)\)であることから以下のような変形を行うことができる。
        \begin{displaymath}
            \begin{split}
                \sum_{j=17}^{21}&P(J = j)(P(T > 21) + P(T < j) - P(j < T \leq 21)\\
                &= \sum_{j=17}^{21}P(J = j)(P(T > 21) + P(T < j) - (1 - P(T > 21) - P(T < j) - P(T = j))\\
                &= \sum_{j=17(}^{21}P(J = j)(2P(T > 21) - 1 + 2P(T < j) + P(T = j))\\
                &= (2P(T > 21) - 1)\sum_{j=17(}^{21}P(J = j) + 2\sum_{j=17(}^{21}P(J = j)P(T < j) + \sum_{j=17(}^{21}P(J = j)P(T = j)\\
                &= (2P(T > 21) - 1)\sum_{j=17(}^{21}P(J = j) + 2P(T < J \leq 21) + P(T = J \leq 21)\\
            \end{split}
        \end{displaymath}

        また、\(P(J < 17)\) + \(\sum_{j=17}^{21}P(J = j) = P( J \leq 21)\)と書けるため、\(E_{d,x}\)は、
        \begin{displaymath}
            \begin{split}
                E_{d,x} &= P(J < 17)(2P(T > 21) - 1) - P(J > 21)\\
                        &\quad+ (2P(T > 21) - 1)\sum_{j=17(}^{21}P(J = j) + 2P(T < J \leq 21) + P(T = J \leq 21)\\
                        &= (2P(T > 21) - 1)(P(J < 17) + \sum_{j=17}^{21}P(J = j)) - P(J > 21)+ 2P(T < J \leq 21) + P(T = J \leq 21)\\
                        &= (2P(T > 21) - 1)P(J \leq 21) - P(J > 21) + 2P(T < J \leq 21) + P(T = J \leq 21)\\
                        &= (2P(T > 21) -1)(1 - P(J > 21)) - P(J > 21) + 2P(T < J \leq 21) + P(T = H \leq 21)\\
            \end{split}
        \end{displaymath}
        となる。したがって、行動決定式\(E_{d,x} - E_{s,x}\)は、
        \begin{displaymath}
            \begin{split}
                E_{d,x} - E_{s,x} &= (2P(T > 21) -1)(1 - P(J > 21)) - P(J > 21) + 2P(T < J \leq 21) + P(T = H \leq 21)\\
                &\quad- ( 2P(T > 21) + 2P(T < x) + P(T = x) - 1)\\
                &= -2P(T < x) - P(T = x) -2P(T > 21)P(J > 21) + 2P(T < J \leq 21) + P(T = J \leq 21)\\
            \end{split}
        \end{displaymath}
        となる。
        \bunseki{※鳥谷 航大}
    \subsection{ベーシックストラテジーの導出}
        ベーシックストラテジーの導出を行う上で、xについて以下の3つの場合分けを行う。
        \begin{eqnarray}
            x<12\\
            12\leq x\leq 16\\
            x>16
        \end{eqnarray}
        \subsubsection{}
            $x<12$のとき、$T \geq 17$であるから、$P(T < x)$と$P(T = x)$は0である。同様に、$P(J > 21)$も0であるから、$E_{d,x} - E_{s,x}$は常に0以上($E_{d,x} - E_{s,x} \geq 0$)となる。よって$x < 12$において、プレイヤーの行動は全てヒットとなる。
        \subsubsection{}
            $12\leq x \leq 16$のとき、1.4.1と同様に$P(T < x)$と$P(T = x)$は0である。この時の行動決定式$E_{d,x} - E_{s,x}$は以下のように書ける。
            \begin{displaymath}
                E_{d,x} - E_{s,x} = -2P(T>21)P(J>21) + \sum_{t-17}^{21}P(T=t)(2P(t<J\leq21) + P(J=t))
            \end{displaymath}
            また、プレイヤーが引いてくるカードの確率は全て同じであるため、
            \begin{eqnarray}
                \begin{split}
                    &P(J-x=10)=4/13 \notag\\
                    &P(J-x=i)=1/13 \quad i=2,3,...,9,(1,11) \notag
                \end{split}
            \end{eqnarray}
            と考えられる。したがって、
            \begin{displaymath}
                \begin{split}
                    &P(J>21)=\frac{1}{13}(x-8)\\
                    &P(T<J\leq21)=\frac{1}{13}(21-T)\\
                    &P(J=T)=\frac{1}{13}\\
                \end{split}
            \end{displaymath}
                とそれぞれ表すことができるため、$E_{d,x} - E_{s,x}$は以下のように書ける。
            \begin{displaymath}
                E_{d,x} - E_{s,x} = -\frac{2}{13}(x-8)P(T>21)+\frac{1}{13}\sum_{t-17}^{21}P(T=t)(43-2t)
            \end{displaymath}
            ここで$x=x_0$として、$E_{d,x_0} - E_{s,x_0}=0$の時を考える。この時、$x_0$は、
            \begin{displaymath}
                \begin{split}
                    x_0&=8+\frac{1}{2}\frac{\sum_{t=17}^{21}(43-2t)P(T=t)}{P(T>21)}\\
                \end{split}
            \end{displaymath}
            となる。このことから、$12\leq x\leq 16$のとき、$M(D)=[x_0]+1$となる。
        \subsubsection{}
            $x>16$のとき、まず、$x=17$を考える。この時$E_{d,x}-E_{s,x}$は、
            \begin{displaymath}
                \begin{split}
                    E_{d,x}-E_{s,x}=-\frac{18}{13}P(T>21)-\frac{5}{13}P(T=17)+\frac{1}{13}\sum_{t=18}^21(43-2t)P(T=t)
                \end{split}
            \end{displaymath}
            となる。この時$E_{d,x}-E_{s,x}$は、全ての$D$に対して常に0未満となる。つまり、$E_{d,x}-E_{s,x}<0$であるから、$M(D)\leq 16$となる。
        
        以上のことから、ベーシックストラテジーは以下のような戦略表となる。
        \begin{table}[h]
        \caption{ベーシックストラテジーの戦略表}
        \centering
        \begin{tabular}{|c|c|c|c|c|c|c|c|c|c|c|}
        \hline
      & 2 & 3 & 4 & 5 & 6 & 7 & 8 & 9 & 10 & A \\ \hline
17以上  & S & S & S & S & S & S & S & S & S  & S \\ \hline
13〜16 & S & S & S & S & S & H & H & H & H  & H \\ \hline
12    & H & H & S & S & S & H & H & H & H  & H \\ \hline
11以下  & H & H & H & H & H & H & H & H & H  & H \\ \hline
        \end{tabular}
        \end{table}
        \bunseki{※鳥谷 航大}
    

%既存戦略の検証に関する項目
\thechapter{プロジェクトの目的}
\thechapter{戦略の検証}
%\section{戦略の検証}

Thorp氏によって考案されたベーシックストラテジーについて検証を行う.
ベーシックストラテジーの性能評価のために比較対象として6つの戦略を考えた.
比較対象となる戦略は次の通りである.

\begin{itemize}
  \item ベーシックストラテジー改変1
  \item ベーシックストラテジー改変2
  \item プレイヤーの合計値が15以上になるまでヒット
  \item プレイヤーの合計値が16以上になるまでヒット
  \item プレイヤーの合計値が17以上になるまでヒット
  \item プレイヤーの合計値が18以上になるまでヒット
\end{itemize}

以上の6つの戦略について,これから一つ一つ詳細に説明する.
\bunseki{※米村祥裕}

\subsection{ベーシックストラテジー改変1}
ベーシックストラテジーはブラックジャックにおける有効な戦略の一つである.しかし
戦略の表に注目すると,表の複雑性を考えたときに変更の余地があると考えた.
戦略表においてプレイヤーの合計値が12の行に注目する.ディーラーのアップカードが2,3の時は
Sとなっているが,その後アップカードが4,5,6の時はH,アップカードが7,8,9,10,Aの時はHとなっており,
Hに挟まれてSが存在している.表を覚えることを考えると,同じ行の中で変化が少ない方がよいと考えられる.
そのため,ベーシックストラテジー改変1では表\ref{bschange1}に示すように,プレイヤーの合計値が12,ディーラーのアップ
カードが4,5,6の時の戦略をSに変更した.
\bunseki{※米村祥裕}

\begin{table}[htbp]
  \centering
  \caption{ベーシックストラテジー改変1\label{bschange1}}
  \begin{tabular}{|c|c|c|c|c|c|c|c|c|c|c|c|}
    \hline
    \multicolumn{2}{|c|}{} & \multicolumn{10}{|c|}{ディーラーのアップカード} \\ \hline
    \multicolumn{2}{|c|}{} & 2 & 3 & 4 & 5 & 6 & 7 & 8 & 9 & 10 & A \\ \hline
    手札の合計 & 19以上 & S & S & S & S & S & S & S & S & S & S \\ \cline{3-12}
              & 18 & S & S & S & S & S & S & S & S & S & S \\ \cline{3-12}
              & 17 & S & S & S & S & S & S & S & S & S & S \\ \cline{3-12}
              & 16 & S & S & S & S & S & H & H & H & H & H \\ \cline{3-12}
              & 15 & S & S & S & S & S & H & H & H & H & H \\ \cline{3-12}
              & 13\~ 14 & S & S & S & S & S & H & H & H & H & H \\ \cline{3-12}
              & 12 & S & S & S & S & S & H & H & H & H & H \\ \cline{3-12}
              & 11以下 & H & H & H & H & H & H & H & H & H & H \\ \hline
  \end{tabular}
\end{table}

\subsection{ベーシックストラテジー改変2}
ベーシックストラテジー改変1と同様にベーシックストラテジーの戦略表を改変した.
変更点は改変1と同様にプレイヤーの合計値が12の行である.改変1ではディーラーのアップカドが4,5,6の部分を
Sに変更したが,改変2ではHに変更した.この変更によってプレイヤーの合計値が12の行はすべてHという表\ref{bschange2}ができた.
\bunseki{※米村祥裕}

\begin{table}[htbp]
  \centering
  \caption{ベーシックストラテジー改変2\label{bschange2}}
  \begin{tabular}{|c|c|c|c|c|c|c|c|c|c|c|c|}
    \hline
    \multicolumn{2}{|c|}{} & \multicolumn{10}{|c|}{ディーラーのアップカード} \\ \hline
    \multicolumn{2}{|c|}{} & 2 & 3 & 4 & 5 & 6 & 7 & 8 & 9 & 10 & A \\ \hline
    手札の合計 & 19以上 & S & S & S & S & S & S & S & S & S & S \\ \cline{3-12}
              & 18 & S & S & S & S & S & S & S & S & S & S \\ \cline{3-12}
              & 17 & S & S & S & S & S & S & S & S & S & S \\ \cline{3-12}
              & 16 & S & S & S & S & S & H & H & H & H & H \\ \cline{3-12}
              & 15 & S & S & S & S & S & H & H & H & H & H \\ \cline{3-12}
              & 13\~ 14 & S & S & S & S & S & H & H & H & H & H \\ \cline{3-12}
              & 12 & H & H & H & H & H & H & H & H & H & H \\ \cline{3-12}
              & 11以下 & H & H & H & H & H & H & H & H & H & H \\ \hline
  \end{tabular}
\end{table}

\subsection{プレイヤーの合計値が15以上になるまでヒット}
ディーラーがある程度有利な戦略を採用しているという仮定の下で,プレイヤーもディーラーと同様の
行動を行う戦略を考えた.プレイヤーの手札合計値が15以上になるまでヒットする戦略の戦略表は\ref{hitleq15}
になる.
\bunseki{※米村祥裕}
\begin{table}[htbp]
  \centering
  \caption{プレイヤーの合計値が15以上になるまでヒット\label{hitleq15}}
  \begin{tabular}{|c|c|c|c|c|c|c|c|c|c|c|c|}
    \hline
    \multicolumn{2}{|c|}{} & \multicolumn{10}{|c|}{ディーラーのアップカード} \\ \hline
    \multicolumn{2}{|c|}{} & 2 & 3 & 4 & 5 & 6 & 7 & 8 & 9 & 10 & A \\ \hline
    手札の合計 & 19以上 & S & S & S & S & S & S & S & S & S & S \\ \cline{3-12}
              & 18 & S & S & S & S & S & S & S & S & S & S \\ \cline{3-12}
              & 17 & S & S & S & S & S & S & S & S & S & S \\ \cline{3-12}
              & 16 & S & S & S & S & S & S & S & S & S & S \\ \cline{3-12}
              & 15 & S & S & S & S & S & S & S & S & S & S \\ \cline{3-12}
              & 13\~ 14 & H & H & H & H & H & H & H & H & H & H \\ \cline{3-12}
              & 12 & H & H & H & H & H & H & H & H & H & H \\ \cline{3-12}
              & 11以下 & H & H & H & H & H & H & H & H & H & H \\ \hline
  \end{tabular}
\end{table}

\subsection{プレイヤーの合計値が16以上になるまでヒット}
プレイヤーの手札合計値が16以上になるまでヒットする戦略の戦略表は\ref{hitleq16}
になる.
\bunseki{※米村祥裕}
\begin{table}[htbp]
  \centering
  \caption{プレイヤーの合計値が16以上になるまでヒット\label{hitleq16}}
  \begin{tabular}{|c|c|c|c|c|c|c|c|c|c|c|c|}
    \hline
    \multicolumn{2}{|c|}{} & \multicolumn{10}{|c|}{ディーラーのアップカード} \\ \hline
    \multicolumn{2}{|c|}{} & 2 & 3 & 4 & 5 & 6 & 7 & 8 & 9 & 10 & A \\ \hline
    手札の合計 & 19以上 & S & S & S & S & S & S & S & S & S & S \\ \cline{3-12}
              & 18 & S & S & S & S & S & S & S & S & S & S \\ \cline{3-12}
              & 17 & S & S & S & S & S & S & S & S & S & S \\ \cline{3-12}
              & 16 & S & S & S & S & S & S & S & S & S & S \\ \cline{3-12}
              & 15 & H & H & H & H & H & H & H & H & H & H \\ \cline{3-12}
              & 13\~ 14 & H & H & H & H & H & H & H & H & H & H \\ \cline{3-12}
              & 12 & H & H & H & H & H & H & H & H & H & H \\ \cline{3-12}
              & 11以下 & H & H & H & H & H & H & H & H & H & H \\ \hline
  \end{tabular}
\end{table}

\subsection{プレイヤーの合計値が17以上になるまでヒット}
プレイヤーの手札合計値が17以上になるまでヒットする戦略の戦略表は\ref{hitleq17}
になる.
\bunseki{※米村祥裕}
\begin{table}[htbp]
  \centering
  \caption{プレイヤーの合計値が17以上になるまでヒット\label{hitleq17}}
  \begin{tabular}{|c|c|c|c|c|c|c|c|c|c|c|c|}
    \hline
    \multicolumn{2}{|c|}{} & \multicolumn{10}{|c|}{ディーラーのアップカード} \\ \hline
    \multicolumn{2}{|c|}{} & 2 & 3 & 4 & 5 & 6 & 7 & 8 & 9 & 10 & A \\ \hline
    手札の合計 & 19以上 & S & S & S & S & S & S & S & S & S & S \\ \cline{3-12}
              & 18 & S & S & S & S & S & S & S & S & S & S \\ \cline{3-12}
              & 17 & S & S & S & S & S & S & S & S & S & S \\ \cline{3-12}
              & 16 & H & H & H & H & H & H & H & H & H & H \\ \cline{3-12}
              & 15 & H & H & H & H & H & H & H & H & H & H \\ \cline{3-12}
              & 13\~ 14 & H & H & H & H & H & H & H & H & H & H \\ \cline{3-12}
              & 12 & H & H & H & H & H & H & H & H & H & H \\ \cline{3-12}
              & 11以下 & H & H & H & H & H & H & H & H & H & H \\ \hline
  \end{tabular}
\end{table}

\subsection{プレイヤーの合計値が18以上になるまでヒット}
プレイヤーの手札合計値が18以上になるまでヒットする戦略の戦略表は\ref{hitleq18}
になる.
\bunseki{※米村祥裕}
\begin{table}[htbp]
  \centering
  \caption{プレイヤーの合計値が18以上になるまでヒット\label{hitleq18}}
  \begin{tabular}{|c|c|c|c|c|c|c|c|c|c|c|c|}
    \hline
    \multicolumn{2}{|c|}{} & \multicolumn{10}{|c|}{ディーラーのアップカード} \\ \hline
    \multicolumn{2}{|c|}{} & 2 & 3 & 4 & 5 & 6 & 7 & 8 & 9 & 10 & A \\ \hline
    手札の合計 & 19以上 & S & S & S & S & S & S & S & S & S & S \\ \cline{3-12}
              & 18 & S & S & S & S & S & S & S & S & S & S \\ \cline{3-12}
              & 17 & H & H & H & H & H & H & H & H & H & H \\ \cline{3-12}
              & 16 & H & H & H & H & H & H & H & H & H & H \\ \cline{3-12}
              & 15 & H & H & H & H & H & H & H & H & H & H \\ \cline{3-12}
              & 13\~ 14 & H & H & H & H & H & H & H & H & H & H \\ \cline{3-12}
              & 12 & H & H & H & H & H & H & H & H & H & H \\ \cline{3-12}
              & 11以下 & H & H & H & H & H & H & H & H & H & H \\ \hline
  \end{tabular}
\end{table}
%\chapter{シミュレータの結果}
\section{結果}
この章では結果について述べる。まず初めに作成したシミュレーターの詳細について述べ、その後シミュレーション結果とその統計結果について言及する。

\subsection{ブラックジャックシミュレーター}
基本戦略とその他の戦略を比較する事を目的に、ブラックジャックのシミュレーターをプログラミング言語(python3)を用いて作成した。このシミュレーターを使用して、基本戦略、基準値15の戦略、基準値16の戦略、基準値17の戦略、基準値18の戦略、基本戦略改変1、基本戦略改変2のそれぞれについて勝利回数、敗北回数、引き分けた回数の3つを調べた。ここでは、シミュレーター内部の詳細を記述していく。

\subsubsection{基本設計}
まず初めに、シミュレーターの基本設計について説明する。今回作成したシミュレーターではブラックジャックを行う際に必要となる要素をクラスとして表現した。具体的にはトランプのカードを表現するカードクラスとそれを一纏めにするデッククラス、ゲーム参加者を表すクラスとそれを継承したプレイヤークラスとディーラークラス、ゲームの勝敗を判定するマネージャークラスのそれぞれを定義した。これらのクラスを用いてブラックジャックのゲームを再現し、基本戦略とその他の戦略を実行するプログラムを作成した。次に各クラスの詳細を記述していく。

\subsubsection{トランプのカードを表現するクラス}
このクラスでは実際のトランプのカードを表現するためにrankという変数にA~Kというトランプのランクを、suitという変数にスペード、ハート、ダイヤ、クラブのスートを定義した。また、J,Q,K,Aの絵札カードは10や11と数える必要があったので、ランクを数字に変換する処理もこちらに書き、valueという変数に入力した。

\subsubsection{デックを表現するクラス}
このクラスでは先程定義したカードクラスを利用してデックを定義した。具体的には先程のカードクラスの配列を作成し、その中にジョーカーを除く52種類のトランプカードを作成した。このクラスの初期化時に使用するデックの数を指定する。また、デックのシャッフルには独自に作成した関数を使用した。このシャッフル関数はpythonのrandom機能を用いて独自に設計したものであり、引数にシャッフルを行う回数を指定する。カードの配列の長さが仮に52だった場合には、1~26番目のカードからランダムに取り出したカードと、27~52番目のからランダムに取り出したカードを交換するという処理を(デック数×指定されたシャッフル回数)繰り返すという処理でシャッフル関数を作成した。

\subsubsection{ゲーム参加者を表すスーパークラス}
このクラスでは自身の手札とその手札の合計値、手札に含まれるAの枚数、バーストしているかどうかのフラグ、手札がブラックジャックとなっているかどうかのフラグのそれぞれを定義している。手札に含まれるAの枚数は自身の手札の合計値を計算する時と、ブラックジャックの条件を満たしているかどうかを判別する際に使用した。また手札の合計値を返す関数を定義し、その内側で自身がバーストしているかどうかの判定も行っている。

\subsubsection{プレイヤークラス}
このクラスは先のゲーム参加者を表すスーパークラスを継承しており、ゲームに参加しているプレイヤーを表現している。プレイヤークラスでは新たに自身の名前を表す変数と自身の勝利回数、敗北回数を記録する変数を定義した。またこのクラスでは新しく、カードを受け取る関数とヒットを行う関数、スタンドを行う関数、勝利回数と敗北回数を増加させる関数を作成した。

\subsubsection{ディーラークラス}
このクラスは先のゲーム参加者を表すスーパークラスを継承しており、ゲームのディーラーを表現しているクラスとなっている。ディーラークラスの中でデックをインスタンス化してディーラー側がデックを所持している事を表現している。このクラスでは新しく、デックのシャッフル回数という変数を定義した。また、このクラスではカードを配る関数、ディーラーの手札合計が17を超えるまでカードを引き続ける関数を作成した。

\subsubsection{ゲームマネージャークラス}
このクラスは主にゲームの勝敗判定に使用している。プレイヤーとディーラーの手札の合計値を比較し勝敗を判定する関数と、手札がブラックジャックになっているかどうかを判定する関数を作成した。勝敗判定のタイミングでプレイヤーの勝利回数、敗北回数のそれぞれを記録している。

\subsubsection{メイン関数}
以上のクラスを用いてメイン関数にブラックジャックのゲームを記述した。以下にプログラムの実行手順を示す。
\begin{enumerate}
    \item ゲームに参加するプレイヤーを作成。今回はプレイヤーを一人のみ作成した。
    \item ディーラーを作成。
    \item カットカードを定義。カットカードを挟む位置はデックの半分の位置とした。
    \item ゲーム全体の実行回数を定義。今回は10万回とした。
    \item プレイヤーの戦略を配列形式で定義した。
    \item ゲームを繰り返すwhile文を作成し、ループ回数を10万回とした。
    \begin{enumerate}
        \item デックからカットカードが出てきたかを確認する。もし出てきていればデックをシャッフルする。
	  \item ディーラーが自身を含む各プレイヤーに初期カードを配る。
	  \item プレイヤーは自身の戦略に沿った行動を選択する。
	  \item すべてのプレイヤーの行動が終了したことを確認後にディーラーが行動を開始する
	  \item ディーラーの行動終了後に、勝敗判定を行う。
    \end{enumerate}
\end{enumerate}

\subsubsection{未実装の機能について}
今回のシミュレーションでは条件を簡単にするためにヒットとスタンドの処理のみを使用しており、その他の複雑なルールについては実装していない。ブラックジャックのすべてのルールをプログラム上に再現できているわけではないので、後期の活動で完全版のシミュレーターを作成する。

\subsection{シミュレーションの条件設定}
今回はデック数1とデック数無限の2つの条件でシミュレーションを行った。共通している条件は以下の通りである。
\begin{itemize}
\item プレイヤー人数は一人
\item 基本戦略、基準値15戦略、基準値16戦略、基準値17戦略、基準値18戦略、基本戦略改変1、基本戦略改変2それぞれの戦略でシミュレーターを実行する
\item ゲームの実行回数は10万回
\end{itemize}

デック数1という条件は元々定義していたデック数の変数を用いて作成した。デック数無限についてはどのカードも引く確率が1/13という事実を利用して、pythonのrandom関数を使用し作成した。具体的にはディーラーがカードを配る関数の部分をデックからカードを引いてくるのではなく、1/13の確率でA~Kいずれかのカードを配る様に設計した。
シミュレーション終了後は勝利回数、敗北回数、引き分け回数の3つをtxt形式で出力した。
\bunseki{※尾崎拓海}

%\section{シミュレータの擬似乱数の検証}
今回シミュレータを作成するにあたり、擬似乱数を使用した。この擬似乱数が妥当かどうかについて検証する。今回使用した擬似乱数生成方法はPythonのrandom()である。このrandom()の擬似乱数を生成するアルゴリズムはメルセンヌツイスタを用いている。そのため今回メルセンヌツイスタについて検証する。
\subsection{周期}
擬似乱数には周期が存在する。周期とは同じ数字が出てくるようになるまでの回数のことを指す。周期が小さいとよく同じ数字が出てきてしまいランダムと言いずらい。逆に周期が大きいと同じ数字が出てきずらくなる。そのため、擬似乱数では周期が大きいほうが性能が良いということになる。メルセンヌツイスタでは周期は$2^{19937}-1$である。これはほかの擬似乱数に比べ、かなり大きい周期であり、メルセンヌツイスタを擬似乱数として使うのに十分であると考えられる。
\bunseki{柿崎大輝}
\subsection{カイ2乗検定}
次はカイ2乗検定を使い、どの値も等しい確率で出ていることを検証する。random()を使用して、0~1の範囲
で1000回値を出す。その後、その値を0~0.1、0.1~0.2、0.2~0.3、0.3~0.4、0.4~0.5、0.5~0.6、0.6~0.7、0.7~0.8、0.8~0.9、0.9~1.0の10通りに分類する。それをまとめると以下の表になる。
\begin{table}[H]
 \begin{center}
  \begin{tabular}{|c|c|c|c|c|c|c|c|c|c|}
    \hline    0~0.1 &  0.1~0.2 & 0.2~0.3 & 0.3~0.4 &  0.4~0.5 & 0.5~0.6 & 0.6~0.7 & 0.7~0.8 & 0.8~0.9 & 0.9~1.0 \\
    \hline 95 & 85 & 100 & 102 & 91 & 114 & 87 & 108 & 115 & 103 \\
    \hline
  \end{tabular}
 \end{center}
 \caption{randomでの結果}
\end{table}
本当にランダムなのであれば、この結果はどれも100になることが予想できる。しかし、実際にはすべてがピッタリ100にはならないので、そのためカイ2乗検定を行い検証する。先ほど出た度数を実現度数として使用し、100を理論度数としてカイ2乗検定を行う。この時、自由度は9で優位水準を5%とすると、棄却値は16.92となり、カイ2乗値がこれより小さいと擬似乱数がどれを等しく出てきたといえる。実際に計算すると、カイ2乗値は9.98となった。この値は16.92より小さいので、擬似乱数によって出た値はすべて等しく出てきたといえる。
\bunseki{柿崎大輝}
\subsection{擬似乱数まとめ}
random()ではメルセンヌツイスタが使われ、かつ周期もカイ2乗検定においても十二分に使えると判断することができるため、今回のシミュレータにおいて使用した。
\bunseki{柿崎大輝}
%\section{シミュレータ結果まとめ}
シュミレーターで10万回ブラックジャックを行った結果を表1に示す。
\begin{table}[H]
 \begin{center}
  \begin{tabular}{|c|c|c|c|c|c|c|}
    \hline
     & \multicolumn{3}{c|}{デック数無限} & \multicolumn{3}{c|}{デック数1} \\
    \cline{2-7} & 勝ち & 負け & 引き分け & 勝ち & 負け & 引き分け \\
    \hline ベーシックストラテジー & 42746 & 48635 & 8619 & 43111 & 48654 & 8235 \\
    \hline ベーシックストラテジー改1 & 42583 & 48782 & 8635 & 42923 & 48909 & 8168 \\
    \hline ベーシックストラテジー改2 & 42223 & 49079 & 8698 & 42955 & 48689 & 8356 \\
    \hline 15以上 & 42392 & 49458 & 8150 & 42063 & 50027 & 7910 \\
    \hline 16以上 & 41410 & 49506 & 9084 & 41580 & 49672 & 8748 \\
    \hline 17以上 & 40870 & 49400 & 9730 & 40974 & 49630 & 9396 \\
    \hline 18以上 & 39291 & 52587 & 8122 & 42071 & 49872 & 8057 \\
    \hline
  \end{tabular}
 \end{center}
 \caption{デック数と各戦略での勝利、負け、引き分け}
\end{table}
今回シミュレータをそれぞれを10万回回して、その結果を勝ち、負け、引き分けの3種類に分けて、表にまとめた。デック数無限の場合、
ベーシックストラテジーが最も勝ちが多く、その後にベーシックストラテジー改1、15以上の戦略の順に勝ちが多い。デック数1の場合、
ベーシックストラテジーが最も勝ちの数が多く、ベーシックストラテジー改1、ベーシックストラテジー改2の順に勝ちが多い。
\section{検定}
先ほどのシミュレータの結果から戦略の間の勝率に有意な差があるかどうかを確かめたい。また、デック数の違いによって戦略の勝率に有意な差があるかも確かめたい。確かめるために検定を行う。
\section{カイ2乗検定}
勝率に有意な差があるかどうかを確かめるためカイ2乗検定を使用する。カイ2乗検定ではカイ2乗値を用いて検定を行う。カイ2乗値は下の式で計算を行う。
\begin{equation} カイ2乗 = \sum{ \frac{(実現度数 - 理論度数)^2}{理論度数}} \end{equation}
実現度数とは実際に出た度数のことで、理論度数は理想としてでる度数のことである。実現度数はシミュレータの結果から参照し、理論度数はシミュレータの結果から計算する。
理論度数は以下の式で計算される。
\begin{equation} 理論度数 =  行の合計 \frac{列の合計}{すべての合計} \end{equation}
\section{戦略間の勝率}
カイ2乗検定を行うためにシミュレータの結果の表を変形する。
\newpage
\begin{table}[H]
 \begin{center}
  \begin{tabular}{|c|c|c|c|}
    \hline
      & 勝ち & 勝ち以外(負けと引き分け) & 合計 \\
    \hline ベーシックストラテジー (実現度数)& 42746 & 57254 & 100000 \\
             ベーシックストラテジー (理論度数)& 41645 & 58355 &  \\
    \hline ベーシックストラテジー改1 (実現度数)& 42583 & 57417 & 100000 \\
             ベーシックストラテジー改1 (理論度数)& 41645 & 58355 &  \\
    \hline ベーシックストラテジー改2 (実現度数)& 42223 & 57777 & 100000 \\
              ベーシックストラテジー改2 (理論度数)& 41645 & 58355 &  \\
    \hline 15以上 (実現度数)& 42392 & 57608 & 100000 \\
             15以上 (理論度数)& 41645 & 58355 &  \\
    \hline 16以上 (実現度数)& 41410 & 58590 & 100000 \\
             16以上 (理論度数)& 41645 & 58355 &  \\
    \hline 17以上 (実現度数)& 40870 & 59130 & 100000 \\
             17以上 (理論度数)& 41645 & 58355 &  \\
    \hline 18以上 (実現度数)& 39291 & 60709 & 100000 \\
             18以上 (理論度数)& 41645 & 58355 &  \\
    \hline  合計 & 291515 & 408485 & 700000 \\
    \hline
  \end{tabular}
 \end{center}
 \caption{デック数無限の場合}
\end{table}
\begin{table}[H]
 \begin{center}
  \begin{tabular}{|c|c|c|c|}
    \hline
      & 勝ち & 勝ち以外(負けと引き分け) & 合計 \\
    \hline ベーシックストラテジー (実現度数)& 43111 & 56889 & 100000 \\
             ベーシックストラテジー (理論度数)& 42240 & 57760 &  \\
    \hline ベーシックストラテジー改1 (実現度数)& 42923 & 57077 & 100000 \\
             ベーシックストラテジー改1 (理論度数)& 42240 & 57760 &  \\
    \hline ベーシックストラテジー改2 (実現度数)& 42955 & 57045 & 100000 \\
              ベーシックストラテジー改2 (理論度数)& 42240 & 57760 &  \\
    \hline 15以上 (実現度数)& 42063 & 57937 & 100000 \\
             15以上 (理論度数)& 42240 & 57760 &  \\
    \hline 16以上 (実現度数)& 41580 & 58420 & 100000 \\
             16以上 (理論度数)& 42240 & 57760 &  \\
    \hline 17以上 (実現度数)& 40974 & 59026 & 100000 \\
             17以上 (理論度数)& 42240 & 57760 &  \\
    \hline 18以上 (実現度数)& 42071 & 57045 & 100000 \\
             18以上 (理論度数)& 42240 & 57760 &  \\
    \hline  合計 & 295677 & 404323 & 700000 \\
    \hline
  \end{tabular}
 \end{center}
 \caption{デック数1の場合}
\end{table}
\clearpage
カイ2乗検定を行う前に帰無仮説、対立仮説と優位水準を定める。\\
帰無仮説:戦略間の勝率に有意な差がない\\
対立仮説:戦略間の勝率に有意な差がある\\
優位水準を5%として、カイ2乗検定を行う。\\
自由度は(行の数-1)(列の数-1)で求められるので、(7-1)(2-1)で自由度6となる。自由度6かつ優位水準5%の時、棄却値は12.96となる。この棄却値よりもカイ2乗値が大きい場合、帰無仮説を棄却して対立仮説が採択される。
デック数無限の場合、カイ2乗値は377.8012となり、377.8012>12.96となるため、帰無仮説を棄却する。つまり、デック数無限の場合、勝率に有意な差が存在する。\\
デック数1の場合、カイ2乗値は127.8811となり、127.8811>12.96となるため、帰無仮説を棄却する。つまり、デック数1の場合、勝率に有意な差が存在する。
\section{デック数による勝率}
カイ2乗検定を行うための表を作成する。まずがベーシックストラテジーでの表を作成する。
\begin{table}[H]
 \begin{center}
  \begin{tabular}{|c|c|c|c|}
    \hline
      & 勝ち & 勝ち以外(負けと引き分け) & 合計 \\
    \hline デック数無限のベーシックストラテジー (実現度数)& 42798 & 57202 & 100000 \\
            デック数無限のベーシックストラテジー (理論度数)& 42955 & 57046 &  \\
    \hline デック数1のベーシックストラテジー (実現度数)& 43111 & 56889 & 100000 \\
            デック数1のベーシックストラテジー (理論度数)& 42955 & 57046 &  \\
    \hline  合計 & 85909 & 114991 & 200000 \\
    \hline
  \end{tabular}
 \end{center}
 \caption{デック数ごとのベーシックストラテジー}
\end{table}
シミュレータの結果からこのような表をそれぞれの戦略ごとに作る。
カイ2乗検定を行う前に帰無仮説、対立仮説と優位水準を定める。\\
帰無仮説:デック数1と無限の間に勝率に有意な差がない\\
対立仮説:デック数1と無限の間勝率に有意な差がある\\
優位水準を5%として、カイ2乗検定を行う。\\
自由度は1となり、優位水準5%なので、棄却値は3.84となる。この棄却値よりもカイ2乗値が大きい場合、帰無仮説を棄却して対立仮説が採択される。\\
ベーシックストラテジーの場合、カイ2乗値は1.0665となり、1.0665<3.84より帰無仮説を棄却しない。つまり、ベーシックストラテジーでデック数が1と無限では勝率に有意な差はない。
これをそれぞれベーシックストラテジー改1、ベーシックストラテジー改2、15以上、16以上、17以上、18以上について同じように行う。
すべてをまとめると、ベーシックストラテジー改1から順にカイ2乗値は4.6112、3.0753、2.2164、0.5952、0.2237、160.1305となった。この中で棄却値を超えたの
はベーシックストラテジー改1と18以上の場合である。ベーシックストラテジー改2、15以上、16以上と17以上の場合、デック数が1と無限では勝率に有意な差はない、
ベーシックストラテジー改1と18以上の場合、デック数が1と無限では勝率に有意な差はあるという結果である。
\section{残差分析}
カイ2乗検定によって各戦略間の勝率に有意な差が存在することが判明した。しかし、どこに有意な差が存在するのかが分からない。そのためさらに検定を行いどの戦略
に有意な差が存在するのかを探す。有意な差がどこに存在するのかを発見するため残差分析を行う。残差分析では調整済み標準化残差を算出し、調整済み標準化残差が1.96
より大きい場合と-1.96より小さい場合に有意な差があると分かる。調整済み標準化残差は以下の式で計算される。
\begin{equation} 調整済み標準化残差 =  \frac{\frac{実現度数 - 理論度数}{\sqrt{理論度数}}}{(1-行比率)(1-列比率)} \end{equation}
調整済み標準化残差をそれぞれ出したものを表でまとめる。
\begin{table}[H]
 \begin{center}
  \begin{tabular}{|c|c|c|c|c|}
    \hline
     & \multicolumn{2}{c|}{デック数無限} & \multicolumn{2}{c|}{デック数1} \\
    \cline{2-5} & 勝ち & 勝ち以外 & 勝ち & 勝ち以外 \\
    \hline ベーシックストラテジー & 7.63 & -7.63 & 6.02 & -6.02  \\
    \hline ベーシックストラテジー改1 & 6.50 & -6.50 & 4.73 & -4.73  \\
    \hline ベーシックストラテジー改2 & 4.00 & -4.00 & 4.95 & -4.95  \\
    \hline 15以上 & 5.18 & -5.18 & -1.22 & 1.22  \\
    \hline 16以上 & -1.63 & 1.63 & -4.56 & 4.56  \\
    \hline 17以上 & -5.37 & 5.37 & -8.75 & 8.75  \\
    \hline 18以上 & -16.31 & 16.31 & -1.17 & 1.17  \\
    \hline
  \end{tabular}
 \end{center}
 \caption{調整済み標準化残差}
\end{table}
この表の1.92以上と-1.92以下が有意な差とある判断できる。
デック数無限の場合、1.92以上で有意に多いと判断できる戦略はベーシックストラテジー、ベーシックストラテジー改1、ベーシックストラテジー改2、15以上である。反対に-19.2以下は17以上と18以上の場合である。
デック数1の場合、1.92以上で有意に多いと判断できる戦略はベーシックストラテジー、ベーシックストラテジー改1、ベーシックストラテジー改2である。反対に-19.2以下は16以上と17以上の場合である。\\
基本戦略、基本戦略改1、基本戦略改2の3つはデック数に関係なく勝ちが有意に多いことが分かる。逆に、17以上はデック数にかかわりなく勝ちが有意に少ないことが分かる。
\section{多重比較}
残差分析によって全戦略のどこに有意な差があるかが分かった。しかし、戦略と戦略の間に有意な差があるかはわからない。そのため、多重比較を行い、戦略間に有意な差があるかどうかを確認する。\\
fisherの正確確率検定を用いて、戦略間のp値をすべて算出し、p値をholm法で調整を施す。p値が0.05以下の場合、有意な差がある。p値が0.05より大きい場合、有意な差がないとする。それを表にまとめた。
\begin{table}[H]
 \begin{center}
 \small
 \scalebox{0.8}[1.0]{
  \begin{tabular}{|c|c|c|c|c|c|c|}
    \hline
     & \multicolumn{6}{c|}{デック数無限} \\
    \cline{2-7} & ベーシックストラテジー & ベーシックストラテジー改1 & ベーシックストラテジー改2 & 15以上 & 16以上 & 17以上 \\
    \hline ベーシックストラテジー改1 勝率差 & 0.1 &  &  &  &  &   \\
    ベーシックストラテジー改1 p値 & 1.000 &  &  &  &  &    \\
    \hline ベーシックストラテジー改2 勝率差 & 0.5 & 0.4 & & & &  \\
    ベーシックストラテジー改2 p値 & 0.109 & 0.522 & & & &   \\
    \hline 15以上 勝率差 & 0.3 & 0.2 & -0.2 & & &   \\
    15以上 p値 & 0.522 & 1.000 & 1.000 & & &  \\
    \hline 16以上 勝率差 & 1.3 & 1.2 & 0.8 & 1.0 & &  \\
    16以上 p値 & 0 & 0 & 0.002 & 0 & &  \\
    \hline 17以上 勝率差 & 1.8 & 1.8 & 1.3 & 1.5 & 0.5 &  \\
    17以上 p値 & 0 & 0 & 0 & 0 & 0 & \\
    \hline 18以上 勝率差 & 3.4 & 3.3 & 2.9 & 3.1 & 2.1 & 1.6  \\
    18以上 p値 & 0 & 0 & 0 & 0 & 0 & 0  \\
    \hline
  \end{tabular}
 }
 \end{center}
 \caption{デック数無限の多重比較}
\end{table}
\begin{table}[H]
 \begin{center}
 \small
 \scalebox{0.8}[1.0]{
  \begin{tabular}{|c|c|c|c|c|c|c|}
    \hline
     & \multicolumn{6}{c|}{デック数無限} \\
    \cline{2-7} & ベーシックストラテジー & ベーシックストラテジー改1 & ベーシックストラテジー改2 & 15以上 & 16以上 & 17以上 \\
    \hline ベーシックストラテジー改1 勝率差 & 0.2 &  &  &  &  &   \\
    ベーシックストラテジー改1 p値 & 1.000 &  &  &  &  &    \\
    \hline ベーシックストラテジー改2 勝率差 & 0.1 & -0.1 & & & &  \\
    ベーシックストラテジー改2 p値 & 1.000 & 1.000 & & & &   \\
    \hline 15以上 勝率差 & 1.0 & 0.8 & 0.9 & & &   \\
    15以上 p値 & 0 & 1.001 & 0.001 & & &  \\
    \hline 16以上 勝率差 & 1.5 & 1.3 & 1.4 & 0.5 & &  \\
    16以上 p値 & 0 & 0 & 0 & 0.158 & &  \\
    \hline 17以上 勝率差 & 2.1 & 2.1 & 2.0 & 1.1 & 0.6 &  \\
    17以上 p値 & 0 & 0 & 0 & 0 & 0.042 & \\
    \hline 18以上 勝率差 & 1.0 & 1.0 & 0.9 & 0 & -0.5 & 1.1  \\
    18以上 p値 & 0 & 0.001 & 0.001 & 1.000 & 0.158 & 0  \\
    \hline
  \end{tabular}
 }
 \end{center}
 \caption{デック数1の多重比較}
\end{table}

表により、有意な差が存在すると部分、存在しない部分が分かる。デック数が無限とデック数が1の両方で有意な差がない部分はベーシックストラテジーと
ベーシックストラテジー改1、ベーシックストラテジーとベーシックストラテジー改2、ベーシックストラテジー改1とベーシックストラテジー改2という結果になる。
%\section{複雑性を考慮した性能比較とその結果}

本項では、複雑性を考慮した性能比較について、また、その結果について説明する。

\subsection{複雑性の定義について}

より人に扱いやすい戦略を定義する為に、A.N. Kolmogorov氏の『On tables of random numbers』を参考にし、戦略の複雑性を次のように設定した。
まず、戦略の文字列を圧縮する。圧縮の方法は、「連続する文字+連続して文字が出た回数」を合わせたものとした。例として、「HHSSSHHHHH」という10字からなる文字列を圧縮すると、「H2S3H5」となり、圧縮した後の文字列は6字となる。この時、連続して文字が出た回数が2桁になったとしても、ここでは1字として数える。

\subsection{各戦略の複雑性}

この圧縮の方式を各戦略に行い、それぞれの圧縮された後の文字列の長さを元の長さで割ったものを複雑性とした。用意した戦略は次のような8行の配列とし、それぞれの行に圧縮を行った。\\

%\begin{figure}[htbp]
%\begin{center}
%\includegraphics[width=15cm,bb=0 0 602 281]{1.png}
%\end{center}
%\caption{基本戦略の戦略表}
%\label{picture}
%\end{figure}

\begin{table}[H]
\caption{基本戦略の戦略表}
\label{table:data_type}
\begin{center}
\begin{tabular}{llllllllllll}
\hline
			     &              & \multicolumn{10}{c}{ディーラーのアップカード}      \\ \cline{3-12} 
                        &              & 2 & 3 & 4 & 5 & 6 & 7 & 8 & 9 & 10 & A \\ \hline
{手札の合計}  & 19以上        & S & S & S & S & S & S & S & S & S  & S \\
                        & 18          & S & S & S & S & S & S & S & S & S  & S \\
                        & 17          & S & S & S & S & S & S & S & S & S  & S \\
                        & 16          & S & S & S & S & S & H & H & H & H  & H \\
                        & 15          & S & S & S & S & S & H & H & H & H  & H \\
                        & 13$\sim$14  & S & S & S & S & S & H & H & H & H  & H \\
                        & 12          & H & H & S & S & S & H & H & H & H  & H \\
                        & 11以下        & H & H & H & H & H & H & H & H & H  & H
\end{tabular}
\end{center}
\end{table}

今回用意した戦略を、全て圧縮したのが以下の表である。


%\begin{figure}[htbp]
%\begin{center}
%\includegraphics[width=15cm,bb=0 0 602 261]{2.png}
%\end{center}
%\caption{各戦略の圧縮した後の文字列と複雑性}
%\label{picture}
%\end{figure}

\begin{table}[H]
\caption{各戦略の圧縮した後の文字列と複雑性}
\label{table:data_type}
\begin{center}
\begin{tabular}{llll}
戦略           & 圧縮した後の文字列                                                                   & 文字列長 & 複雑性   \\ \hline
基本戦略         & \begin{tabular}[c]{@{}l@{}}S10S5H5S5H5S5H5S5H5S5H5H2S3H5H10\end{tabular} & 30   & 0.375 \\
基本戦略改変1      & S10S5H5S5H5S5H5S5H5S5H5S5H5H10                                              & 28   & 0.35  \\
基本戦略改変2      & S10S5H5S5H5S5H5S5H5S5H5H10H10                                               & 26   & 0.325 \\
15以上になるまでヒット & S10S10S10S10S10H10H10H10                                                    & 16   & 0.2   \\
16以上になるまでヒット & S10S10S10S10H10H10H10H10                                                    & 16   & 0.2   \\
17以上になるまでヒット & S10S10S10H10H10H10H10H10                                                    & 16   & 0.2   \\
18以上になるまでヒット & S10S10H10H10H10H10H10H10                                                    & 16   & 0.2  
\end{tabular}
\end{center}
\end{table}

この表を見ると、基本戦略を改変した戦略の方が、複雑性が低く、人にとって扱いやすいといえる。また、一定の数字以上になるまでヒットを続ける戦略は、複雑性が基本戦略の半分程度であり、とても人にとって扱いやすい戦略だといえる。

\subsection{各戦略の性能評価}

そして、今回はその複雑性を用いて、各戦略の性能比較を行った。
性能の基準は以下の二通りを用意した。\\
~~ 1. (勝率) ÷ (複雑性)\\
~~ 2. (勝率) - (複雑性)\\
この評価基準に従って、1デックの時と無限デックの時の性能を表にした。\\\\\\\\\\\\\\\\\\\\\\\\




%\begin{figure}[htbp]
%\begin{center}
%\includegraphics[width=15cm,bb=0 0 602 279]{3.png}
%\end{center}
%\caption{1デックの時の各戦略の性能}
%\label{picture}
%\end{figure}

\begin{table}[H]
\caption{1デックの時の各戦略の性能}
\label{table:data_type}
\begin{center}
\begin{tabular}{llllll}
戦略           & 圧縮長 & 勝率    & 複雑性   & 性能1  & 性能2   \\ \hline
基本戦略         & 30  & 0.431 & 0.375 & 1.14 & 0.052 \\
基本戦略改変1      & 28  & 0.429 & 0.35  & 1.22 & 0.076 \\
基本戦略改変2      & 26  & 0.430 & 0.325 & 1.21 & 0.072 \\
15以上になるまでヒット & 16  & 0.421 & 0.200 & 2.12 & 0.224 \\
16以上になるまでヒット & 16  & 0.416 & 0.200 & 2.07 & 0.214 \\
17以上になるまでヒット & 16  & 0.410 & 0.200 & 2.05 & 0.209 \\
18以上になるまでヒット & 16  & 0.421 & 0.200 & 1.97 & 0.193
\end{tabular}
\end{center}
\end{table}


%\begin{figure}[htbp]
%\begin{center}
%\includegraphics[width=15cm,bb=0 0 541 255]{4_.png}
%\end{center}
%\caption{無限デックの時の各戦略の性能}
%\label{picture}
%\end{figure}

\begin{table}[H]
\caption{1デックの時の各戦略の性能}
\label{table:data_type}
\begin{center}
\begin{tabular}{llllll}
戦略           & 圧縮長 & 勝率    & 複雑性   & 性能1  & 性能2   \\ \hline
基本戦略         & 30  & 0.427 & 0.375 & 1.14 & 0.052 \\
基本戦略改変1      & 28  & 0.424 & 0.35  & 2.12 & 0.224 \\
基本戦略改変2      & 26  & 0.414 & 0.325 & 1.07 & 0.214 \\
15以上になるまでヒット & 16  & 0.424 & 0.200 & 2.05 & 0.209 \\
16以上になるまでヒット & 16  & 0.414 & 0.200 & 1.97 & 0.193 \\
17以上になるまでヒット & 16  & 0.409 & 0.200 & 1.21 & 0.076 \\
18以上になるまでヒット & 16  & 0.393 & 0.200 & 1.21 & 0.072
\end{tabular}
\end{center}
\end{table}

各戦略を比較し、次のような結果を得た。
まず、勝率のみを考慮した場合、一定の数字以上でスタンドする戦略よりも、基本戦略とそれを改変した戦略の方が有意に高い勝率だった。
また、基本戦略と改変1、改変2のそれぞれの戦略間には有意な差が見られなかった。
複雑性を考慮して性能を評価した場合、基準値を15に設定した戦略が一番優秀であった。

\subsection{考察}

勝率のみを見ると、1デック、無限デック共に基本戦略が最も勝率が高かった。しかし、扱いやすさも含めた性能を評価すると、必ずしも基本戦略が扱いやすいとは限らず、改善の余地があるということが分かった。

\subsection{今後の課題}

前期のプロジェクト学習におけるブラックジャックの前提では、それぞれのゲームは1ゲームで行われており、過去に出たゲームが次以降のゲームに影響されることはなかった。そのため、有限のデックで連続したゲームを行った場合の戦略について考える必要がある。

また、今回は勝率のみを考えた場合を想定していた。実際のゲームでは、賭け金の概念があるので、それを導入した場合にどのように利得をプラスにするか、そのための戦略を考える必要がある。それに伴い、今回のプロジェクトでは省いたダブルダウン、スプリット、サレンダー等のルールを含めて最終的な利得をプラスにする戦略を考えたい。

戦略の扱いやすさについて、今回は複雑性の設定を手動で行い、検証する時間もあまり取らなかったので、評価基準が正確ではない可能性がある。
今後、この評価基準をどのように調整するかも検討の余地がある。

\bunseki{※渡邊凛}

\thechapter{検証結果}
\thechapter{前期活動}
\thechapter{後期活動}
\thechapter{中間発表の評価}
\input{./evaluation.tex}

\begin{thebibliography}{9}
  \bibitem{blackjack1} 齋藤隆浩 (1999) 『新訂ブラックジャック必勝法』, 株式会社データハウス
  \bibitem{blakjack2} Roger Baldwin, Wilbert Cantey, Herbert Maisel, James McDermott (1956) "The Optimum Strategy in Blackjack", Journal of the American Statistical Association, 51:275, 419-439
  \bibitem{neuro1} 萩原将文 (1994) 『ニューロ・ファジィ・遺伝的アルゴリズム』, 産業図書株式会社
  \bibitem{neuro2} 斎藤康毅 (2016) 『ゼロから作るDeep Learning Pythonで学ぶディープラーニングの理論と実装』, 森北出版株式会社
  \bibitem{pattern1} Christopher M. Bishop(2007)『Pattern Recognition and Machine Learning』(元田浩 他 訳 (2016) 『パターン認識と機械学習 上』 , 丸善出版)
  \bibitem{ga1} 棟朝雅晴 (2008) 『遺伝的アルゴリズム その理論と先端的手法』, 森北出版株式会社
  \bibitem{ga2} 北野宏明 他 (1993) 『遺伝的アルゴリズム』, 産業図書株式会社
  \bibitem{complexity} Buenos Aires, Argenlina (1969) 『On the Simplicity and Speed of Programs for Computing Infinite Sets of Natural Numbers』 Journal of the Association for Computing Machinery, Vol.16,No.3,July 1969,pp. 407-422
  \bibitem{statistics1} オライリー・ジャパン山内光哉(1987) 『心理・教育のための統計法』, 株式会社サイエンス社
\end{thebibliography}


%暫定削除予定
%\input{./neuralnetwork.tex}
%\input{./parameter.tex}
%\input{./neuro2.tex}
%\input{./geneticalgorithm.tex}


\end{document}
