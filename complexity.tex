\section{複雑性を考慮した性能比較とその結果}

本項では、複雑性を考慮した性能比較について、また、その結果について説明する。

\subsection{複雑性の定義について}

より人に扱いやすい戦略を定義する為に、A.N. Kolmogorov氏の『On tables of random numbers』を参考にし、戦略の複雑性を次のように設定した。
まず、戦略の文字列を圧縮する。圧縮の方法は、「連続する文字+連続して文字が出た回数」を合わせたものとした。例として、「HHSSSHHHHH」という10字からなる文字列を圧縮すると、「H2S3H5」となり、圧縮した後の文字列は6字となる。この時、連続して文字が出た回数が2桁になったとしても、ここでは1字として数える。

\subsection{各戦略の複雑性}

この圧縮の方式を各戦略に行い、それぞれの圧縮された後の文字列の長さを元の長さで割ったものを複雑性とした。用意した戦略は次のような8行の配列とし、それぞれの行に圧縮を行った。\\

%\begin{figure}[htbp]
%\begin{center}
%\includegraphics[width=15cm,bb=0 0 602 281]{1.png}
%\end{center}
%\caption{基本戦略の戦略表}
%\label{picture}
%\end{figure}

\begin{table}[H]
\caption{基本戦略の戦略表}
\label{table:data_type}
\begin{center}
\begin{tabular}{llllllllllll}
\hline
			     &              & \multicolumn{10}{c}{ディーラーのアップカード}      \\ \cline{3-12} 
                        &              & 2 & 3 & 4 & 5 & 6 & 7 & 8 & 9 & 10 & A \\ \hline
{手札の合計}  & 19以上        & S & S & S & S & S & S & S & S & S  & S \\
                        & 18          & S & S & S & S & S & S & S & S & S  & S \\
                        & 17          & S & S & S & S & S & S & S & S & S  & S \\
                        & 16          & S & S & S & S & S & H & H & H & H  & H \\
                        & 15          & S & S & S & S & S & H & H & H & H  & H \\
                        & 13$\sim$14  & S & S & S & S & S & H & H & H & H  & H \\
                        & 12          & H & H & S & S & S & H & H & H & H  & H \\
                        & 11以下        & H & H & H & H & H & H & H & H & H  & H
\end{tabular}
\end{center}
\end{table}

今回用意した戦略を、全て圧縮したのが以下の表である。


%\begin{figure}[htbp]
%\begin{center}
%\includegraphics[width=15cm,bb=0 0 602 261]{2.png}
%\end{center}
%\caption{各戦略の圧縮した後の文字列と複雑性}
%\label{picture}
%\end{figure}

\begin{table}[H]
\caption{各戦略の圧縮した後の文字列と複雑性}
\label{table:data_type}
\begin{center}
\begin{tabular}{llll}
戦略           & 圧縮した後の文字列                                                                   & 文字列長 & 複雑性   \\ \hline
基本戦略         & \begin{tabular}[c]{@{}l@{}}S10S5H5S5H5S5H5S5H5S5H5H2S3H5H10\end{tabular} & 30   & 0.375 \\
基本戦略改変1      & S10S5H5S5H5S5H5S5H5S5H5S5H5H10                                              & 28   & 0.35  \\
基本戦略改変2      & S10S5H5S5H5S5H5S5H5S5H5H10H10                                               & 26   & 0.325 \\
15以上になるまでヒット & S10S10S10S10S10H10H10H10                                                    & 16   & 0.2   \\
16以上になるまでヒット & S10S10S10S10H10H10H10H10                                                    & 16   & 0.2   \\
17以上になるまでヒット & S10S10S10H10H10H10H10H10                                                    & 16   & 0.2   \\
18以上になるまでヒット & S10S10H10H10H10H10H10H10                                                    & 16   & 0.2  
\end{tabular}
\end{center}
\end{table}

この表を見ると、基本戦略を改変した戦略の方が、複雑性が低く、人にとって扱いやすいといえる。また、一定の数字以上になるまでヒットを続ける戦略は、複雑性が基本戦略の半分程度であり、とても人にとって扱いやすい戦略だといえる。

\subsection{各戦略の性能評価}

そして、今回はその複雑性を用いて、各戦略の性能比較を行った。
性能の基準は以下の二通りを用意した。\\
~~ 1. (勝率) ÷ (複雑性)\\
~~ 2. (勝率) - (複雑性)\\
この評価基準に従って、1デックの時と無限デックの時の性能を表にした。\\\\\\\\\\\\\\\\\\\\\\\\




%\begin{figure}[htbp]
%\begin{center}
%\includegraphics[width=15cm,bb=0 0 602 279]{3.png}
%\end{center}
%\caption{1デックの時の各戦略の性能}
%\label{picture}
%\end{figure}

\begin{table}[H]
\caption{1デックの時の各戦略の性能}
\label{table:data_type}
\begin{center}
\begin{tabular}{llllll}
戦略           & 圧縮長 & 勝率    & 複雑性   & 性能1  & 性能2   \\ \hline
基本戦略         & 30  & 0.431 & 0.375 & 1.14 & 0.052 \\
基本戦略改変1      & 28  & 0.429 & 0.35  & 1.22 & 0.076 \\
基本戦略改変2      & 26  & 0.430 & 0.325 & 1.21 & 0.072 \\
15以上になるまでヒット & 16  & 0.421 & 0.200 & 2.12 & 0.224 \\
16以上になるまでヒット & 16  & 0.416 & 0.200 & 2.07 & 0.214 \\
17以上になるまでヒット & 16  & 0.410 & 0.200 & 2.05 & 0.209 \\
18以上になるまでヒット & 16  & 0.421 & 0.200 & 1.97 & 0.193
\end{tabular}
\end{center}
\end{table}


%\begin{figure}[htbp]
%\begin{center}
%\includegraphics[width=15cm,bb=0 0 541 255]{4_.png}
%\end{center}
%\caption{無限デックの時の各戦略の性能}
%\label{picture}
%\end{figure}

\begin{table}[H]
\caption{1デックの時の各戦略の性能}
\label{table:data_type}
\begin{center}
\begin{tabular}{llllll}
戦略           & 圧縮長 & 勝率    & 複雑性   & 性能1  & 性能2   \\ \hline
基本戦略         & 30  & 0.427 & 0.375 & 1.14 & 0.052 \\
基本戦略改変1      & 28  & 0.424 & 0.35  & 2.12 & 0.224 \\
基本戦略改変2      & 26  & 0.414 & 0.325 & 1.07 & 0.214 \\
15以上になるまでヒット & 16  & 0.424 & 0.200 & 2.05 & 0.209 \\
16以上になるまでヒット & 16  & 0.414 & 0.200 & 1.97 & 0.193 \\
17以上になるまでヒット & 16  & 0.409 & 0.200 & 1.21 & 0.076 \\
18以上になるまでヒット & 16  & 0.393 & 0.200 & 1.21 & 0.072
\end{tabular}
\end{center}
\end{table}

各戦略を比較し、次のような結果を得た。
まず、勝率のみを考慮した場合、一定の数字以上でスタンドする戦略よりも、基本戦略とそれを改変した戦略の方が有意に高い勝率だった。
また、基本戦略と改変1、改変2のそれぞれの戦略間には有意な差が見られなかった。
複雑性を考慮して性能を評価した場合、基準値を15に設定した戦略が一番優秀であった。

\subsection{考察}

勝率のみを見ると、1デック、無限デック共に基本戦略が最も勝率が高かった。しかし、扱いやすさも含めた性能を評価すると、必ずしも基本戦略が扱いやすいとは限らず、改善の余地があるということが分かった。

\subsection{今後の課題}

前期のプロジェクト学習におけるブラックジャックの前提では、それぞれのゲームは1ゲームで行われており、過去に出たゲームが次以降のゲームに影響されることはなかった。そのため、有限のデックで連続したゲームを行った場合の戦略について考える必要がある。

また、今回は勝率のみを考えた場合を想定していた。実際のゲームでは、賭け金の概念があるので、それを導入した場合にどのように利得をプラスにするか、そのための戦略を考える必要がある。それに伴い、今回のプロジェクトでは省いたダブルダウン、スプリット、サレンダー等のルールを含めて最終的な利得をプラスにする戦略を考えたい。

戦略の扱いやすさについて、今回は複雑性の設定を手動で行い、検証する時間もあまり取らなかったので、評価基準が正確ではない可能性がある。
今後、この評価基準をどのように調整するかも検討の余地がある。

\bunseki{※渡邊凛}
