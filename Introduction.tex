\section{背景}
この章では、ブラックジャックの歴史やそれについての戦略、本プロジェクトの目的を紹介する。また、それらについて、考えられる課題を明らかにする。
\bunseki{※菱田美紗紀}

\subsection{ブラックジャックの概要と歴史}
ブラックジャックのルーツは1570年にさかのぼる。このころはまだ「ブラックジャック」という名称は使われていなかった。1875年に出版された
「The American Hoyle of 1875」という書籍で「ブラックジャック」として紹介されたのが初出である。このゲームが考え出された当初は金銭
を賭けることはなく、身内内で楽しむだけのゲームであった。はじめて金銭が賭けられるようになったのは1910年頃のアメリカ、インディアナ州
であるといわれている。当時は競馬以外のギャンブルが違法であり、ブラックジャックも違法カジノでのみ行われていた。その後インディアナ州
以外のアメリカ全土に広がっていった。そして現在世界中のカジノで合法的に楽しまれるポピュラーなゲームになった。
\bunseki{※菱田美紗紀}

\subsection{ブラックジャックの戦略の歴史}
ブラックジャックについては多くの人間が戦略を考え出しては、カジノ側に禁止をされてきた歴史がある。1950年に、メリーランド州のとある米国
陸軍の研究所に所属していた、Roger Nash Baldwin氏らが研究し始めたのが、ブラックジャックの戦略の研究のはじまりであるといわれている。
その後パソコンが発明されたことにより、シミュレーションが容易になったことでさらに戦略の研究は進んでいった。ブラックジャックには主に有名
な戦略が2つ存在する。ベーシックストラテジーと呼ばれる、ディーラーのアップカードと自分の手札によって戦略を決定するものと、カウンティング
と呼ばれる、今まで出たカードを記憶して戦略を決定するものである。まず、ベーシックストラテジーについて詳しく説明する。この戦略は、1962年、
カリフォルニア大学アーバイン校の教授である、Edward Thorp氏が、書籍『Beat The Dealer:A Winning Strategy for the Game of Twenty One』
にて発表した。この戦略は、ディーラーのアップカード、自分の初期の手札の2つの情報から、次に自分がどのような行動を行えばよいのか決定される戦
略である。この戦略を使用すれば勝率は4割~5割弱を出すことができる。カウンティングについては後期に詳しく調査する予定である。
なお、いずれの戦略もカジノでは禁止されており、これらの戦略を使用していると気づかれたとき、プレイヤーはカジノから追放される。
\bunseki{※菱田美紗紀}